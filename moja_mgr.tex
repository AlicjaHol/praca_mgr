%
% Szablon, v. 3.0
% p.wlaz@pollub.pl
%

% PROSZĘ NIE USUWAĆ
% KOMENTARZY Z~PREAMBUŁY
% JEŻELI KTOŚ WAM WMAWIA, ŻE
% TO PRZYSPIESZY COKOLWIEK
% -- MYLI SIĘ!


\documentclass[12pt]{mwbk}


%%%%%%% marinesy, rozmiary, to warto dopasować do drukarki
\usepackage[a4paper,twoside,top=2.6cm,bottom=2.6cm,inner=3cm,outer=2.6cm]{geometry}

%%%%%%%% polszczyzna
\usepackage[T1]{polski}


%%%%%%%%% sposób kodowania literek w edytorze
\usepackage[utf8]{inputenc}

\usepackage[font=small,labelfont=bf,justification=centering]{caption}


%%%% gdyby ktoś chciał powyklejać z~pedeefa
%%%% teksty za pomocą AcroReadera, to 
%%%% poniższe dwie linijki pomogą w~tym
%%%% Może to być przydatne, gdyby ktoś na podstawie
%%%% elektronicznej wersji chciał przygotować dane do 
%%%% badania antyplagiatowego
%%%% ponieważ prace są w
%%%% tych czasach różnymi
%%%% programami antyplagiatowymi
%%%% proszę absolutni NIE
%%%% USUWAĆ następujących
%%%% dwu linijek
\input glyphtounicode.tex
\pdfgentounicode = 1

%%%%%%%%%%%%%%%%%%%%%%%%%%%%%%%%%%%%%
%%%%% jeśli chcesz by główny tekst oraz wzory matematyczne były
%%%%% składane czcionką typu Times Roman (w~odróżnieniu od standardowej
%%%%% TeXowej, czyli Computer Modern Roman) to linia poniżej
%%%%% ma być 'aktywna', następna nieaktywna, 
%%%%% jeśli zrobisz odwrotnie (pierwsza nieaktywna,
%%%%% druga aktywna) uzyskasz skład czcionką
%%%%% Computer Modern Roman mającą wielu wiernych
%%%%% fanów w~świecie TeXa). Konsekwencją jednak będą zmiany
%%%%% rozmiarów czcionek dla rozdziałó i podrozdziałów - rzecz bez większego
%%%%% znaczenia, wynikająca z pewnych zaszłości historycznych (ComputerModern
%%%%% niegdyś były używane wyłącznie w postaci tzw. bitmap)
\usepackage{mathptmx} \usepackage{tgtermes}
%\usepackage{lmodern}

%%% WSZELKIE ZMIANY W~PREAMBULE RÓB ROZWAŻNIE
%%% NIE JESTEŚ PEWNY/PEWNA ICH EFEKTU TO~SPRAWDŹ 
%%% CZY W~PRGRAMIE ADOBE READER (i~to dokładnie
%%% o~ten program chodzi, nie o~jakikolwiek)
%%% Z~WYNIKOWEGO PLIU PDF DA SIE PRAWIDŁOWO
%%% WYKLEIĆ TEKST Z~POLSKIMI LITERAMI, BEZ KRZAKÓW,
%%%% BEZ DZIWACTW.


%%%%%%%%%%%%%% pozostałe pakiety używane w~pracy, to już zależy od
%%%%%%%%%%%%%% autora, więc może być tego więcej
\usepackage{fancyhdr}
\usepackage{graphicx}
\usepackage{amsmath}
\usepackage{amsthm}
\usepackage{amssymb}
\usepackage{url}
\usepackage{longtable}
\usepackage{array,hhline}

%%%%%%%% hyperref po to by przeglądarka pedeef ukazywala na odwołania
%%%%%%%% prawidłowo skonstruowane za pomocą \ref, \cite i.t.d. jako
%%%%%%%% hiperłącza
\usepackage{hyperref}



%%%%% dla fanów ``profesjonalnych'' tabel w~stylu zachodnich książek

\usepackage{booktabs} \heavyrulewidth=1.5bp \lightrulewidth=0.5bp


%%%%%%%%%%% poniżej uniwersalny sposób na ucywilizowanie znaków 
%%%%%%%%%%% niewiększości, niezależny od pakietu {polski}, ale za to 
%%%%%%%%%%% zależny od {amssymb}, ma tą zaletę, że działa np. z Timesem
%%%%%%%%%%% w matematyce
\let\leq\leqslant\let\le\leq\let\geq\geqslant\let\ge\geq


%%%%%%% jeżeli będziesz chciał włączać do swojej pracy fragmenty programów, 
%% to ponizsza linijka przyda się, jeśli nie - usuń ją

\usepackage{fancyvrb}


%%%%%%%%%%%%%%%%% struktury do tworzenia twierdzeń i~tym podobnych

\theoremstyle{plain}
\newtheorem{twier}{Twierdzenie}[chapter] % pierwsze to nazwa środowiska,
                                      %drugie to wyświetlana nazwa
				% to trzecie w~nawiasie kwadratowym
				% wskazuje numer dolepiony z~lewej do
				% numeru twierdzenia (tu numer
				% 'chapter', 
\newtheorem{lemat}{Lemat}[chapter]

\theoremstyle{definition}
\newtheorem{defi}{Definicja}[chapter]

\theoremstyle{remark}
\newtheorem{uwaga}{Uwaga}[chapter]
\newtheorem{wniosek}{Wniosek}[chapter]

%%%%% więcej możliwości w~dokumentacji amsthm



%%%%%%%%%%%%%%%%%%%%%%%%%%%%%%%%%%%%%%%%%5
%%%%%%%%%%%%%%%%%%%%%%%%%%%%%%%%%%%%%%%%%%
%%%%%%%%% wcięcie akapitowe %%%%%%%%%%%%%%
%%%%%%%%%%%%%%%%%%%%%%%%%%%%%%%%%%%%%%%%%%
%%%%%% ustawić w~zaleceń i~gustu %%%%%%%%%
%%%%%%%%%%%%%%%%%%%%%%%%%%%%%%%%%%%%%%%%%%
%%%%%%%% zalecenie na stronie wydziałowej
%%%%%%%% było 1.25cm i wyglądało jakoś 
%%%%%%%% śmiesznie duże, więc spłoszony nieco
%%%%%%%% wpisałem 1cm, ale uważny czytelnik już
%%%%%%%% zapewne się domyśli, że podmiana napisu 
%%%%%%%% =1cm na =1.25cm sprawi, że wcięcia na początku
%%%%%%%% akapitu ustawią się na (nieco przydużą)
%%%%%%%% wartość 1.25cm 

\parindent=1cm



%%%%%%%%%%%%%%%%%%%%%%%%%%%%%%%%
%%%%% tu pewne poluzowanie rozmieszczenia elementów tabelek
%%%%% możecie sobie poeksperymentować, by dopasować do swych
%%%%% gustów, a przede wszystkim gustów promotorów (promotorek)
  \tabcolsep=4mm          
  %\renewcommand\arraystretch{1.3}
%%%%%%%%%%%%%%%%%%%%%%%%%%%%%%%%%%



%%%%%%%%% teraz żywa pagina (aka 'running headline') i~numerowanie stron
%%%%%%%%%%%%%%%%%%%%%%%%%%%%%%%%%%%%%%%%%%%%%%%%%%%%%%%%%%%%%%%%%%%%%%%%
%%%%%na górze mają być śródtytuły, na dole (po stronie zewneętrznej)
%%%%%numery stron. Poszedłem kapkę dalej i~na stronach ropoczynających
%%%%%rozdział nie ma paginy (górki).
%%%%% Oczywiście jeśli ostatnia strona
%%%%% jest pusta (uzupełnia jeno parzystość) to tam żadnej stopki ani 
%%%%% górki byc mnie może - ma być pusta.
%%%%%%%%%%%%%%%%%%%%%%%%%%%%
\pagestyle{fancy}
\fancyhead{}% oczyszczenie
\fancyhead[RO]{\rightmark} %% na nieparzystych 'podległe' śródtytuły
\fancyhead[LE]{\leftmark} %% na parzystych 'ważniejsze'
\fancyfoot{}% oczyszczenie
\fancyfoot[RO,LE]{\arabic{page}}  %% numer na dole (po prawej na
%% nieparzystych, po lewej na parzystych)
\renewcommand\headrulewidth{0.4pt} %%% cienka hrulka oddzielająca paginę
                                    %%% od kolumny tekstu
\fancypagestyle{closing}{%%%%%% to styl dla stron zamykających rozdział
\fancyhead{}% oczyszczenie
\fancyhead[RO]{\rightmark} %% na nieparzystych 'podległe'
\fancyhead[LE]{\leftmark} %% na parzystych 'ważniejsze'
\fancyfoot{}% oczyszczenie
\fancyfoot[RO,LE]{\arabic{page}}  %% numer na dole (po prawej na
                                  %% powyższą linijkę usuń jeśli nie
				  %% chcesz numerów na niepełnych
				  %% kolumnach (zamykających rozdział)
\renewcommand\headrulewidth{0.4pt}
}
\fancypagestyle{opening}{%%% styl stron rozpoczynających rozdział
\fancyhead{}% oczyszczenie
\fancyfoot{}% oczyszczenie
\fancyfoot[RO,LE]{\arabic{page}}  %% numer na dole (po prawej na
\renewcommand\headrulewidth{0pt}
}
\fancypagestyle{plain}{%%%% styl zwykły, niektóre konstrukcje
                       %%%% (typu \titlepage, którego ja tu nie używam
                       %%%% ale może są jakieś inne o których nawet nie chce 
                       %%% mi się myśleć, więc dla spokoju robię to po swojemu
\fancyhead{}% oczyszczenie
\fancyfoot{}% oczyszczenie
\fancyfoot[RO,LE]{\arabic{page}}  %% numer na dole (po prawej na
\renewcommand\headrulewidth{0pt}
}

%%%%%%%%%%%%%%%%%%%%%%%%%%%%%%%%%5
%%%%%%%%%%%%%%%%%%%%%%%%%%%%%%%%%%
%%% lekka modyfikcja 'markow' do paginy
%%% uznalem, ze jesli ktos nie da \section (np we wstepnie czy
%%% podsumowaniu to niech na obu sronach w~paginie pojawia sie tytuł
%%% chaptera, bo standardowo, to na nieparzystej stronie w takiej sytuacji
%%% nad górną linią ziałaby pustka, co mogłoby wprowadzać konsternację
\makeatletter
    \def\chaptermark#1{%
      \markboth{%
        \ifHeadingNumbered
     \if@mainmatter
     \@chapapp\
            \thechapter.\enspace
          \fi
        \fi
        #1}{%
        \ifHeadingNumbered
     \if@mainmatter
     \@chapapp\
            \thechapter.\enspace
          \fi
        \fi
        #1%
	}}%
    \def\sectionmark#1{%
      \markright{%
        \ifHeadingNumbered \thesection.\enspace \fi
        #1}}
%%%%%%%%%%%%%%%%%%%%%%%%%%%%%%%%%%%%%%%%%%%%%%%
%%%%%%%%%%%%%%%%%%%%%%%%%%%%%%%%%%%%%%%%%%%%%%%%
%%%%%%%%%%%% wielkości czcionek dla chapter i~section
%%%%%%%%%%%% 16 dla rozdziału, 14 dla podrozdziału - te domyślne
%%%%%%%%%%%% w klasie mwbk były całkiem ładne, ale żeby nie było
%%%%%%%%%%%% że nie potrafię ustawić
%%%%%%%%%%%%%%%%%%%%%%%%%%%%%%%%%%%%%%%%%%%%%%%%%%%
\SetSectionFormatting[breakbefore,wholewidth]{chapter}
        {0\p@}
        {\FormatRigidChapterHeading{6.4\baselineskip}{12\p@}%
	{\large\@chapapp\space}{\fontsize{16}{19}\selectfont}}
        {1.6\baselineskip}
\SetSectionFormatting{section}
        {24\p@\@plus5\p@\@minus2\p@}
	{\FormatHangHeading{\fontsize{14}{16}\selectfont}}
        {10\p@\@plus3\p@}
\makeatother	



%%%%%%%%%%%%%%%%%%%%%%%%%%%%%%%%%%%%%%%%%%%%%%
%%%%%%%%%%%%%%%%%%%%%%%%%%%%%%%%%%%%%%%%%%%%%%
%%%%%%%%%%%%%% jakies inne pomocnicze definicje, ja na przykład lubię
% \R
%%%%%%%%%%%%%%%%%%%%%%%5
%%%%%%%%%%%%%%%%%%%%%%%
%%%% tak naprawdę są t potrzebne tylko po to
%%%% by zadziałały przykłady poniżej w tekście
%%%% które w sposób dość losowy zostały 
%%%% pobrane z jakichś moich starych plików
%%%%%%%%%%%%%%%%%%%%%%%%%%%%%%%%%%
%%%%%%%%%%%%%%%%%%%%%%%%%%%%%%%%%%%
%%%% w realnej pracy te poniższe śmieci możecie oczywiście
%%%% usunąć
%%%%%%%%%%%%%%%%%%%%%%%%%%%%
\newcommand\R{\mathbb{R}}
\newcommand{\ff}{\mathbf{f}}
\newcommand{\hh}{\mathbf{h}}
\newcommand{\xx}{\mathbf{x}}
\newcommand{\yy}{\mathbf{y}}
\newcommand{\zz}{\mathbf{z}}
\newcommand{\gggg}{\mathbf{g}}
\newcommand{\skalar}[2]{\pmb{\langle}#1,#2\pmb{\rangle}}
%%%%%%%%%%%% koniec tych dodatkowych definicji

%%%%%% trocę więcej ``luzu'' przy rozmieszczaniu {fgur} i~{table}

 \renewcommand{\topfraction}{0.9}	% max fraction of floats at top
    \renewcommand{\bottomfraction}{0.8}	% max fraction of floats at bottom
    %   Parameters for TEXT pages (not float pages):
    \setcounter{topnumber}{2}
    \setcounter{bottomnumber}{2}
    \setcounter{totalnumber}{4}     % 2 may work better
    \setcounter{dbltopnumber}{2}    % for 2-column pages
    \renewcommand{\dbltopfraction}{0.9}	% fit big float above 2-col. text
    \renewcommand{\textfraction}{0.07}	% allow minimal text w. figs
    %   Parameters for FLOAT pages (not text pages):
    \renewcommand{\floatpagefraction}{0.7}	% require fuller float pages
    % N.B.: floatpagefraction MUST be less than topfraction !!
    \renewcommand{\dblfloatpagefraction}{0.7}	% require fuller float pages
    % remember to use [htp] or [htpb] for placement

    
%%% DWA proste polecenia służące do ujednolicenia podawania źródeł przy rysunkach i~tabelkach    
    
    \newcommand\zrodlo[1]{\par\vspace{-3mm}{\small\textit{Źródło: }#1 }}
    \newcommand\zrodlotab[1]{{\par\vspace{2mm}\small\textit{Źródło: }#1 }}

\raggedbottom   %%% to znaczy, że nie będzie siłowego wyrównywania typowych
                %%     stron do jednakowej wysokości

\linespread{1.3}
\begin{document}

%%%%%%%%%%%%%%%%%%%%%%%%%%%%%%%%%%%%%%%%%
%%%%%%%%%%%%%%%%%%%%%%%%%%%%%%%%%%%%%%%%%
%%%%%%%% STRONA TYTUŁOWA %%%%%%%%%%%%%%%%

\thispagestyle{empty}  % tu wszak nie chcemy żadnej numeracji stron


%%%%%%%%%%%%%%%%%%%%%%%%%%%%%%%%%%%%%%%%%%%%%%%%%%%%%%%%%%%%%%%
%%%%%tytuły definiuje jako makrodefinicje, gdyż zamierzam je%%%
%%%%%powtórzyć na stronie ze streszczeniami, to nic nie boli%%%
%%%%%a gwarantuje, że będą one takie same, i~tak ma być.%%%%%%%
%%%%%%%%%%%%%%%%%%%%%%%%%%%%%%%%%%%%%%%%%%%%%%%%%%%%%%%%%%%%%%%
\newcommand\tytul{Przegląd autoenkoderów stosowanych w nienadzorowanym uczeniu maszynowym}

\newcommand\tytulangielski{An overview of autoencoders used in unsupervised machine learning}



\noindent\hspace{-32pt}\includegraphics{rys/logopl}
\begin{center}
	



%%{\large \bf POLITECHNIKA LUBELSKA}

%%%% {\bf WYDZIAŁ PODSTAW TECHNIKI} tego już nie chcą w~nowym wzorcu karty

%%%% \emph{Kierunek:} MATEMATYKA   %% już jest w~``logo''

%%% BEZ SPEC.!!! \emph{Specjalność:} Matematyka w~finansach i~ubezpieczeniach

\vfill %%%% \vfill to taki rozpychacz w pionie, pcha ile mu pozwolą
     


\vfill

\textbf{Praca magisterska}

\vfill
\vfill
\vfill

\large
\tytul

\vfill

\emph{\tytulangielski}


\vfill
\vfill
\vfill
\vfill
\vfill

\begin{tabular}[t]{l}
\emph{Praca wykonana pod kierunkiem:}
\\
dra Dariusza Majerka
\end{tabular}
\hfill
\begin{tabular}[t]{l}
	\emph{Autor:}
\\
Alicja Hołowiecka\\
nr albumu: 89892 
\end{tabular}

\vfill
\vfill
\vfill

\textbf{Lublin 2022}

\end{center}


%%%%% koniec tytułów


%%%%%%%%%%%%%%%%%%%%%%%5
%%%%%%%%%%%%%%%%%%%%%%
%%% teraz spis treści
%%%%%%%%%%%%%%%%%%%%%
%%% pamiętaj! po jakiejkolwiek zmianie w tekście
%%% która wpływa na zmianę spisu treści, spis będzie dobry co najmniej
%%% po dwóch przebiegach latexa - to samo dotyczy odwołań do wzorów i literatury
%%% ogólnie to przed wydrukiem warto przelatexować o jedne raz więcej niż
%%% to się wydaje konieczne, no chyba że korzystamy z funkcji typu BUILD
%%% w zintegrowanym systemie wspomagającym TeX, BUILD powinien takie sprawy 
%%% wziąć pod uwagę

\tableofcontents


\chapter*{Wstęp}



\chapter{Przegląd autoenkoderów}

\section{Sztuczne sieci neuronowe}

\textbf{Sztuczny neuron} ma co najmniej jedno binarne wejście i dokładnie jedno binarne wyjście. Wyjście jest uaktywniane, jeżeli jest aktywna określona liczba wejść. Na rysunku \ref{fig:neurony1} przedstawione są przykładowe sztuczne sieci neuronowe (SSN lub z angielskiego ANN) wykonujące różne operacje logiczne, przy założeniu, że neuron uaktywni się, gdy przynajmniej dwa wejścia będą aktywne \cite{geron}.

\begin{figure}[!h]
	\centering
	\includegraphics[width=9cm]{rys/neurony1.png}
	\caption{Przykładowe sztuczne sieci neuronowe rozwiązujące proste zadania logiczne}
	\zrodlo{\cite{geron}}
	\label{fig:neurony1}
\end{figure}

Jedną z najprostszych architektur SSN jest \textbf{perceptron}, którego podstawą jest sztuczny neuron zwany \textbf{progową jednostką logiczną} (ang. \textit{Threshold Logic Unit} - TLU) lub \textbf{liniową jednostką progową} (ang. \textit{Linear Threshold Unit} - LTU). Wartościami wejść i wyjść są liczby, a każde połączenie ma przyporządkowaną wagę. Jednostka TLU oblicza ważoną sumę sygnałów wejściowych, a następnie zostaje użyta funkcja skokowa. Schemat takiej jednostki został przedstawiony na rysunku \ref{fig:neurony2}.

\begin{figure}[!h]
	\centering
	\includegraphics[width=8cm]{rys/neurony2.png}
	\caption{Struktura sztucznego neuronu, który stosuje funkcję skokową $f$ na ważonej sumie sygnałów wejściowych}
	\zrodlo{\cite{ertel}}
	\label{fig:neurony2}
\end{figure}

Często używaną funkcją skokową jest \textbf{funkcja Heaviside'a}, określona równaniem 

$$H(z)=\begin{cases}
0, & \text{ jeśli } z<0\\
1, & \text{ jeśli } z \geq 0
\end{cases}$$

Czasami stosuje się również \textbf{funkcję signum}

$$sgn(z)=\begin{cases}
-1 & \text{ jeśli } z<0\\
0, & \text{ jeśli } z=0\\
1, & \text{ jeśli } z > 0
\end{cases}$$

Perceptron jest złożony z jednej warstwy jednostek TLU, w której każdy neuron jest połączony ze wszystkimi wejściami. Tego typu warstwa jest nazywana \textbf{warstwą gęstą}. Warstwa, do której są dostarczane dane wejściowe, jest nazywana \textbf{warstwą wejściową} (ang. \textit{input layer}). Najczęściej do tej warstwy jest wstawiany również \textbf{neuron obciążeniowy} (ang. \textit{bias neuron}) $x_0=1$, który zawsze wysyła wartość 1. Na rysunku \ref{fig:perceptron1} znajduje się perceptron z dwoma neuronami wejściowymi i jednym obciążeniowym, a także z trzema neuronami w warstwie wyjściowej.

\begin{figure}[!h]
	\centering
	\includegraphics[width=10cm]{rys/perceptron1.png}
	\caption{Perceptron z trzema neuronami wejściowymi i trzema wyjściami}
	\zrodlo{\cite{geron}}
	\label{fig:perceptron1}
\end{figure}

Obliczanie sygnałów wyjściowych w warstwie gęstej przedstawia się wzorem
$$h_{\mathbf{W},\mathbf{b}}(\mathbf{X})=\phi(\mathbf{XW}+\mathbf{b})$$
gdzie $\mathbf{X}$ - macierz cech wejściowych, $\mathbf{W}$ - macierz wag połączeń (oprócz neuronu obciążeniowego), $\mathbf{b}$ - wektor obciążeń zawierający wagi połączeń neuronu obciążeniowego ze wszystkimi innymi neuronami, $\phi$ - tzw. \textbf{funkcja aktywacji}, w przypadku TLU jest to funkcja skokowa.

Algorytm uczący, który służy do trenowania perceptronu, jest silnie inspirowany działaniem neuronu biologicznego. Gdy biologiczny neuron często pobudza inną komórkę nerwową, to połączenia między nimi stają się silniejsze. Reguła ta jest nazywana \textbf{regułą Hebba}. Perceptrony są uczone za pomocą odmiany tej reguły, w której połączenia są wzmacniane, jeśli pomagają zmniejszyć wartość błędu. Dokładniej, w danym momencie perceptron przetwarza jeden przykład uczący i wylicza dla niego predykcję. Na każdy neuron wyjściowy odpowiadający za nieprawidłową prognozę następuje zwiększenie wag połączeń ze wszystkimi wejściami przyczyniającymi się do właściwej prognozy. Aktualizowanie wag przedstawia się następującym wzorem
$$\Delta w_{ij}=\eta (y_j-\hat{y_j})x_i$$
gdzie $w_{ij}$ - waga połączenia między $i$-tym neuronem wejściowym a $j$-tym neuronem wyjściowym, $x_i$ - $i$-ta wartość wejściowa bieżącego przykładu uczącego, $\hat{y_j}$ - wynik $j$-tego neuronu wyjściowego dla bieżącego przykładu uczącego, $y_j$ - docelowy wynik $j$-tego neuronu, $\eta$ - współczynnik uczenia.

Perceptron ma wiele wad związanych z niemożnością rozwiązania pewnych trywialnych problemów (np. zadanie klasyfikacji rozłącznej czyli XOR). Część tych ograniczeń można wyeliminować, stosując architekturę SSN złożoną z wielu warstw perceptronów, czyli \textbf{perceptron wielowarstwowy} (ang. \emph{Multi-Layer Perceptron}). Składa się on z jednej warstwy wejściowej (przechodniej), co najmniej jednej warstwy jednostek TLU - tzw. \textbf{warstwy ukryte} (ang. \emph{latent layers}) i ostatniej warstwy jednostek TLU - warstwy wyjściowej. Oprócz warstwy wejściowej każda warstwa zawiera neuron obciążający i jest w pełni połączona z następną wartswą. Sieć zawierająca wiele warstw ukrytych nazywamy \textbf{głęboką siecią neuronową} (ang. \emph{Deep Neural Network} - DNN). 

Do uczenia perceptronów wielowarstwowych wykorzystywany jest algorytm \textbf{propagacji wstecznej} (ang. \emph{backpropagation}). Propagacja wsteczna jest właściwie algorytmem gradientu prostego \cite{skansi}. Można go zapisać jako
$$w_{updated}=w_{old}-\eta \nabla E$$
gdzie $E$ jest funkcją kosztu (funkcją straty) \cite{skansi}. Proces jest powtarzany do momentu uzyskania zbieżności z rozwiązaniem, a każdy przebieg jest nazywany \textbf{epoką} (ang. \emph{epoch}).

\begin{uwaga}
	Wagi połączeń wszystkich warstw ukrytych należy koniecznie zainicjować losowo. W przeciwnym przypadku proces uczenia zakończy się niepowodzeniem. Na przykład jeśli wszystkie wagi i obciążenia zostaną zainicjowane wartością 0, to model będzie działał tak, jak gdyby składał się tylko z jednego neuronu. Przy zainicjowaniu wag losowo, symetria zostanie złamana i algorytm propagacji wstecznej będzie w stanie wytrenować zespół zróżnicowanych neuronów \cite{geron}. 
\end{uwaga}

Aby algorytm propagacji wstecznej działał prawidłowo, kluczową zmianą jest zastąpienie funkcji skokowej przez inne \textbf{funkcje aktywacji}. Zmiana ta jest konieczna, ponieważ funkcja skokowa zawiera jedynie płaskie segmenty i przez to nie pozwala korzystać z gradientu. 

Najczęściej używana jest \textbf{funkcja logistyczna (sigmoidalna)}
$$\sigma(z)=\frac{1}{1+e^{-z}}$$
Ma ona w każdym punkcie zdefiniowaną pochodną niezerową, dzięki czemu algorytm gradientu prostego może na każdym etapie uzyskać lepsze wyniki. Zbiór wartości tej funkcji wynosi od 0 do 1.

Inną popularną funkcją aktywacji jest \textbf{tangens hiperboliczny}
$$\operatorname{tanh}(z)=2\sigma(2z)-1$$
Funkcja ta jest ciągła i różniczkowalna, a jej zakres wartości wynosi $-1$ do 1. Dzięki temu zakresowi wartości wynik każdej warstwy jest wyśrodkowany wobec zera na początku uczenia, co często pomaga w szybszym uzyskaniu zbieżności.

Wśród popularnych funkcji aktywacji należy także wyróżnić  \textbf{funkcję ReLU} (ang. \emph{Rectified Linear Unit} - prostowana jednostka liniowa) o wzorze
$$ReLU(z)=max(0,z).$$
Jest ona ciągła, ale nieróżniczkowalna w punkcie 0. Jej pochodna dla $z<0$ wynosi zero. Jej atutem jest szybkość przetwarzania. Nie ma ona maksymalnej wartości wyjściowej.
W zadaniach regresji bywa wykorzystywany ,,wygładzony'' wariant funkcji ReLU, czyli funkcja \textbf{softplus}:
$$softplus(z)=log(1+exp(z))$$

Na  rysunku \ref{fig:funkcje-aktywacji} przedstawiono popularne funkcje aktywacji wraz z ich pochodnymi.

\begin{figure}[!h]
	\centering
	\includegraphics[width=\linewidth]{rys/funkcje_aktywacji.png}
	\caption{Przykładowe funkcje aktywacji wraz z pochodnymi}
	\zrodlo{\cite{geron}}
	\label{fig:funkcje-aktywacji}
\end{figure}

\section{Sieci splotowe}

Sieci splotowe, nazywane również \textbf{splotowymi sieciami neuronowymi} (ang. \emph{convolutional neural networks}, CNN), są rodzajem sieci neuronowych służących do przetwarzania danych o znanej topologii siatki. Przykładem takich danych są szeregi czasowe, które można uznać za jednowymiarową siatkę z próbkami w regularnych odstępach czasu, oraz dane graficzne, które można interpretować jako dwuwymiarową siatkę pikseli. Nazwa sieci splotowych pochodzi od wykorzystywanego przez te sieci działania matematycznego nazywanego \textbf{splotem} (konwolucją). Można powiedzieć, że sieci splotowe to po prostu sieci neuronowe, które w przynajmniej jednej z warstw zamiast ogólnego mnożenia macierzy wykorzystują splot \cite{goodfellow}.

\section{Czym jest autoenkoder}

	\textbf{Autoenkoder} (czasem także nazywany autokoderem, z ang. \textit{autoencoder}, \textit{auto-encoder}) jest specjalnym typem sieci neuronowej, która jest przeznaczona głównie do kodowania danych wejściowych do skompresowanej i znaczącej reprezentacji, a następnie dekodowania ich z powrotem w taki sposób, aby zrekonstruowane dane były jak najbardziej podobne do oryginalnych \cite{bank}. Autoenkodery uczą się gęstych reprezentacji danych, tzw. reprezentacji ukrytych (ang. \emph{latent representations}) lub kodowań (ang. \emph{codings}) bez jakiejkolwiek formy nadzorowania (tzn. zbiór danych nie zawiera etykiet). Wyjściowe kodowania zazwyczaj mają mniejszą wymiarowość od danych wejściowych, dzięki czemu autoenkodery mogą z powodzeniem służyć do redukcji wymiarowości. Mają też zastosowanie w \textbf{modelach generatywnych} (ang. \emph{generative models}), które potrafią losowo generować nowe dane przypominające zbiór uczący. Jeszcze lepszej jakości dane można uzyskać przy użyciu \textbf{generatywnych sieci przeciwstawnych}, czyli GAN (ang. \textit{Generative Adversial Networks}). Sieci GAN są często stosowane w zadaniach zwiększania rozdzielczości obrazu, koloryzowania, rozbudowanego edytowania zdjęć, przekształcania prostych szkiców w realistyczne obrazy, dogenerowywania danych służących do uczenia innych modeli, generowania innych typów danych np. tekstowych, dźwiękowych, itd. \cite{geron}.


\chapter{Przykłady zastosowań autoenkoderów}

\chapter*{Podsumowanie i~wnioski}


\begin{thebibliography}{99}

\bibitem{aggarwal} Aggarwal, C. C. (2018).\emph{ Neural Networks and Deep Learning}, Springer International Publishing

\bibitem{bank} Bank D., Koenigstein N., Giryes R. (2021), \emph{Autoencoders}, arXiv:2003.05991v2

\bibitem{chollet} Chollet F., Allaire J. J. (2019) \emph{Deep Learning. Praca z językiem R i biblioteką Keras}, Helion SA

\bibitem{edureka} Edureka, Autoencoders Tutorial. Autoencoders in Deep Learning. Tensorflow Training \url{https://www.youtube.com/watch?v=nTt_ajul8NY} 

\bibitem{ertel} Ertel W. (2017) \emph{Introduction to Artificial Intelligence. Second Edition}, Springer International Publishing

\bibitem{geron} G\'eron A. (2020) \emph{Uczenie maszynowe z użyciem Scikit-Learn i TensorFlow. Wydanie II}, Helion SA

\bibitem{goodfellow} Goodfellow I., Bengio Y., Courville A. (2018), \emph{Deep Learning. Systemy uczące się}, PWN, Warszawa 

\bibitem{osinga} Osinga D. (2019) \emph{Deep Learning. Receptury}, Helion SA

\bibitem{skansi} Skansi S. (2018) \emph{Introduction to Deep Learning. From Logical Calculus to Artificial Intelligence}, Springer International Publishing


\end{thebibliography}



\listoffigures

\listoftables


\chapter*{Załączniki}
\begin{enumerate}
%\item Oświadczenie o oryginalności pracy i możliwości jej wykorzystania. 
%\item Opinia promotora na temat oryginalności pracy oraz w~sprawie dopuszczenia do obrony pracy dyplomowej.
%\item Potwierdzenie analizy antyplagiatowej.
\item Płyta CD z niniejszą pracą w wersji elektronicznej.
\end{enumerate}




\chapter*{Streszczenie (Summary)}

\bigskip
\bigskip

\begin{center}
  \textbf{\tytul}
\end{center}



\bigskip

\begin{center}
  \textbf{\textit{\tytulangielski}}
\end{center}



\selecthyphenation{english}
{\it

}

\end{document}

